\chapter{Abstract}

In these unprecedented times, rapid shifts throughout society are forcing businesses and individuals to be readily positioned to adapt, innovate, and transform their ways of being. Largely chaperoned by the constant advances in technology, these shifting expectations and opportunities impose indiscriminate impacts on the organisations that surround us. For those experiencing homelessness, these surrounding organisations will often consist of a consortium of not-for-profit (NFP) organisations, of whom the readiness to acclimatise to the changing world is of critical importance. We set out to investigate the technological intersection between these NFPs, and the individuals whom they altruistically set out to support. Our understanding builds initially on the currently understood technology utilisation patterns of these two parties in isolation, before ultimately discussing and developing the augmentations that technology can facilitate in connecting them. In developing and delivering a real-world example of a technology-based connection strategy in an established Australian NFP, we bring forth our deepened understanding and findings of the space, as well as propose a number of potential future studies to extend upon our work. The findings put forth in this report aim to assist not just further research, but also the practical implementations of technology in homelessness sector NFPs. In brief, we find that a number of considerations must be made by NFPs in the delivery of new technologies, specifically the implications of the \emph{network effect}, the benefits of relying on existing interpersonal relationships, and the limitations on using a shared device. We also recommend the utilisation of existing design principles due to the benefits of familiarity, and finally we see that the desire to share a recording of one's voice extends beyond that of simple stories. NFPs may benefit from this report through our detailed recommendations on designing, delivering, and exploring technology within their service models.