\chapter{Literature Review}

\section{Foundations}

Technology has undeniably reshaped almost every facet of our modern world. Indiscriminately of an individual's socio-economic background, or a business' operating industry, the method in which almost any conceivable task can be completed will now include some form of technological augmentation. Whilst these augmentations may seem to be universal, there are still a number of business sectors and groups of people who have not been a constituent in this global shift. Whether it be business sectors that are unwilling or otherwise unable to embrace this change, or individuals who have inadequate access, there does ultimately a number of parties that stand to benefit from their unrealised technological potential.

This review of the literature intends to examine the position of two specific parties, amongst the wider technological adaptation of the modern world. Specifically, the two parties are those individuals experiencing homelessness, and the NFP organisations that aim to assist them. We will first examine the role of technology within both of these parties, when considered in isolation. Before ultimately delving into the ways in which technology has, does, and should augment the connection between the two. The aim of this review is to understand potential opportunities for technology to be utilised such that the homelessness NFP sector, inclusive of both the organisations and those utilising the services, may be greater connected.

In an effort to ensure currency and relevancy of our research, we have primarily focused on information from recent years, given the immense velocity of technological change. Historical information has also been used in order to provide deeper context or understanding, but was not considered when drawing conclusions about the current state. We also primarily focused on research relating specifically to NFPs supporting homelessness, however we also recognise that many areas of NFPs are agnostic of the sector in which they are directly supporting, such as fundraising, volunteering, and community engagement. Where appropriate, we have considered NFP research which extends beyond the homelessness sector, in order to gain a better understanding of the NFP space.

The specific sources which were considered for this review are almost entirely from published, reputable, academic sources, along with a small sampling of sources from other publications and media as was considered relevant. This sampling of non-academic sources was only considered when it came from a reputable organisation deemed to be in a relevant sector. News outlets and opinion pieces were not considered for this review.

\section{Technology and the Not-for-Profit Sector}

From as early as the 1980s, researchers were astutely aware that emerging technology could provide a \emph{“huge competitive leap”} that would be \emph{“open to exploitation by others”} if not appropriately capitalised on \cite{benjamin1983information}. And as the world progresses even deeper into the Information Age, it is important for organisations to continuously adapt to the evolving technological landscape. This revolution provides opportunity, for those who have the access and resources to evolve; and challenge, for those who do not.

In an effort to better categorise and understand the different levels of technological adoption a business may have attained, Haines proposes a model of four distinct levels of web service adoption \cite{haines2003levels}. Haines suggests that there are not only systematic differences between the levels of adoption, but also between the organisational requirements to reach each level, and ultimately the potential business outcomes. The four levels are defined as \emph{Technical Solution}, \emph{IT Solution}, \emph{Internal Business Solution}, and \emph{External Business Solution}. Haines also goes on to relate these levels to the broader Information Systems innovation types proposed earlier by Swanson \cite{swanson1994information}. Understanding the implications of these categorisations gives strong insight into the important factors to be considered when reviewing the technological adoption of an organisation, with this review focusing specifically on NFPs.

NFPs are precariously poised in this technological shift, as the benefits of technology stand to drastically reduce operational overhead, simplify incoming funding, simplify data collection, and enhance reporting \cite{boles2013technology} \cite{bopp2017disempowered}. However, the capabilities required to utilise these benefits are potentially not within reach of these organisations.

A recent report on technology utilisation in the Australian NFP sector found that although NFPs spent similar amounts on technology to equivalent small to medium size businesses, only 4\% of these NFPs consider themselves to be \emph{“highly satisfied”} with the way they utilise technology \cite{infoxchange_2019}. The report further outlines some underlying correlations which may be contributing to this lack of satisfaction. 56\% of NFP staff members were either \emph{“not confident”} or only \emph{“a bit confident”} in using technology and systems, and this was generally not well alleviated by the NFPs with 40\% not providing staff with any development opportunities in these areas. From a branding perspective, the report did find that most NFPs do have a good understanding of the importance of their externally facing digital footprint, with website and social media improvements emerging as the top priorities across all participant organisations.

The difficulties and challenges faced by NFPs in trying to embrace technology were further investigated in another recent study \cite{le2008view}. The study found that a variety of factors within NFPs may result in the inability to fully realise the potential technological benefits, more specifically: \emph{growing pains in rapidly scaling NFPs}, \emph{lack of clear organisational structure}, and \emph{insufficient computer literacy}. The Infoxchange report identified similar challenges, with the most common across all participant organisations being \emph{budget/financing}, \emph{staff capacity and capability}.

The challenges that have been observed in NFPs, and the subsequent limitation in technological adoption, align well with the Haines' model discussed earlier. Haines identified that for an organisation to achieve a level of adoption higher than \emph{IT Solution}, the organisation is required to have organisation-wide scope in their technology strategy, be willing to undergo significant changes in internal structures and process, and champion innovation from an administrative level. Evidently, the observed challenges conflict directly with these requirements, and ultimately result in an NFPs inability to enhance their level of adoption without first striving to meet Haines' requirements.

Not only are NFPs aiming to augment their existing operations, fundraising activities, and reporting methods through innovative use of technology, they are also aiming to expand their activities beyond what was previously possible. For example, Ko \& Liu recently investigated the emerging trend for NFPs to extend beyond their traditional revenue streams, and explore more direct commercial opportunities as social enterprises \cite{ko2020transformation}. One of these potential commercial opportunities for NFPs is direct to consumer retail purchases, delivered either physically or online. A 2020 Australia Post report, reflecting on the impacts of COVID-19 on eCommerce, showed a staggering 95\% year-over-year growth in online shopping across the country, highlighting a potentially lucrative opportunity for technologically advanced NFPs to capitalise on \cite{auspost2020}.

Whilst there are evidently many possible ways in which NFPs can benefit from technology, it has been found that there are few NFPs that fully realise these benefits. The potential for NFPs to identify and execute on technological opportunities can be greatly enhanced by further investigation and understanding of the ways in which technology is represented in their wider community.

For NFPs aiming to positively impact the lives of those experiencing homelessness, the individuals experiencing homelessness themselves form the most critical technological user group in their wider community.

\subsection{Traditional Technology Strategy}

NFPs, like any other type of business, can vary immensely in their composition. From massive multinational NFPs like the Red Cross or UNICEF, all the way down to grassroots organisations comprising just one individual with a mission. The Australian Charities and Not-for-profits Commission (ACNC) classifies almost two-thirds of the nations NFPs as \emph{“small”}, with annual revenues below \$250,000; at the other end of the spectrum, the ACNC recognises only 0.4\% of charities in its highest classification, as those who have an annual reported revenue over \$100M \cite{acnc2020}. This immense variation in scale can make it difficult in attempting to analyse and aggregate the needs, challenges, and opportunities of NFPs in their entirety. Although NFPs do vary immensely in their current state, there can be considered a shared path that NFPs take to get to where they are today.

Work by Srinivasan, in studying a number of emerging and rapidly growing NFPs, ultimately identified a number of consistencies in the growth patterns and life cycles of NFPs \cite{srinivasan2007understanding}. Specifically, the work proposes that it is common for NFPs to progress through the fundamental stages of \emph{Start-up}, \emph{Expansion/Growth}, \emph{Consolidation}, \emph{Phase-out/Metamorphosis}. The observations made in this work acknowledge that the complexities arise due to the underlying tendency for NFPs to be project organisations at their core. Whilst this work does not explicitly make reference to the role of technology throughout these stages, it follows that the need for enhanced levels of adoption would simultaneously increase, in alignment with the outlined monotonically increasing sophistication across the organisation more generally. It has also been observed that during the \emph{Start-up} stage of an NFP's life cycle, there is potential to grow rapidly through a crowdfunding model \cite{paschen2017choose}, a model which has been shown to be increasingly effective when supported by technology \cite{polishchuk2019technology}.

Prior to the current boom in accessibility of technology in the wider business landscape, it was often seen as too expensive, or unfit, for the needs of an NFP \cite{infoxchange_2019} \cite{corder2001acquiring} \cite{le2008view}. Alongside this perceived inaccessibility, there was also traditionally a steep learning curve required to fully recognise the benefits of technology within an organisation \cite{mcwilliams1996time} \cite{lai2017literature}. Prior to the ubiquity of modern software and devices, proficiency in certain technologies was often a desirable and expensive skill for someone to possess, and therefore often beyond the affordability of NFPs.

\subsection{Modern Technology Strategy}

Evolution in the ways in which technology is delivered to, and utilised by, organisations has seen a drastic shift in the ability to scale. Software solutions, once seen as a tool for the largest and wealthiest of organisations, is now readily available thanks to the growth of the Software as a Service (SaaS) industry. Recent analysis shows that seen SaaS spend has increased almost 50\% in the last two years, across a selection of progressive organisations \cite{blissfully2020}. The ease with which an organisation can discover and implement new technologies to augment their processes enables these organisations to scale readily and rapidly. It is now extremely common for almost all organisations, including NFPs, to have a set of SaaS tools underpinning almost everything they do. Not only are these tools easier to discover, and more intuitive to use, but they are also usually far more affordable \cite{ma2008pricing}.

The need for specialised skills and knowledge to properly utilise technology within an organisation is all but gone, outside the most niche cases; with the ability to adequately operate SaaS applications has become almost as ubiquitous as the ability to type was in generations past \cite{garrido2010understanding} \cite{fischer2005computational} \cite{jackson2010international}. Major business benefits can be realised on the back of this underlying expectation of fundamental technical competencies. This expectation, particularly on the employees of an NFP, enables modern tools to be integrated readily across the board; ultimately enabling the NFP to spend more of their time and resources on their core business, rather than on any process that can be automated, streamlined, or otherwise augmented by technology. More specifically, NFPs stand to benefit from this revolution in technology accessibility through a variety of avenues \cite{kobelsky2014impact}.

In relation to an NFPs core business, technology is able to assist in delivering their service more reliably, safely, and at a larger scale \cite{boles2013technology}. These enhancements in service delivery allow for an NFP to not only provide a better service to their end users, but also to provide a service which is readily able to scale and assist in the growth of the organisation. Technology is also capable of assisting NFPs in algorithmically optimisation in matching the capacity of their service delivery models with the needs and distribution of those experiencing homelessness, as was outlined by Dombrowski in addressing issues faced by a geographically distributed food service \cite{dombrowski2013takes}.

NFPs also stand to benefit in the fundraising space, as these new tools will assist not only in payment processing, but also in running campaigns, tracking donors over time, and numerous other enhancements that can be obtained through the recent advances in eCommerce \cite{goecks2008charitable} \cite{boeder2002non}. The ability to reach and engage a wider audience is obviously also benefited through technology, by integrating a social media strategy into an NFPs wider brand strategy there is near limitless potential in the diversity and shear scale of individuals that can be reached \cite{lovejoy2012information} \cite{nah2013modeling} \cite{wyllie2016examination} \cite{milde2017strategies}.

Alongside these benefits specifically pertaining to NFPs, there are countless other opportunities for modern tools to be beneficial to the organisation; with everything from payroll and regulatory compliance \cite{mahajan2015review}, through to instant messaging and website content management. Although it is obvious to see the upsides when considering how an NFP may utilise these tools, it also must be considered the potential downsides, or barriers to entry, when it comes to the real world implementation with an NFP.

\subsection{Barriers to Progressive Technology Strategies}

The eradication of numerous barriers that traditionally existed to adequately utilising technology, whilst almost total, has not entirely resulted in the removal of all such challenges.

For example, NFPs in the \emph{Start-up} stage of their growth, likely to be in a position of greater agility and comfortability when it comes to change, are in a far better position to embrace this technological shift when compared to an NFP in the \emph{Consolidation}, or later, stages. One of the most universal barriers to a large transition toward technology comes through the fundamental issues of managing change within an organisation \cite{o2007towards} \cite{schein1994management}. An organisation with more ingrained and entangled operations may find it more difficult to entirely transition away from the processes that have, in some cases, been in place for decades or more. Research shows that there are mechanisms and ideologies that can be employed by an organisation to mitigate this barrier of organisational change, Markus suggests that implementing change incrementally can \emph{“sharply cut the risks of technochange”} \cite{markus2004technochange}, and alternatively emphasises the importance of guiding and supporting the individuals through their unlearning of old systems and processes \cite{becker2010facilitating}.

On the other hand, \emph{Start-up} stage NFPs may be willing to adopt technology, but may not have the strategy and clarity to do so effectively. Without strong clarity on the machinations of an organisation, both currently and looking towards the future, may result in inefficient forays into the technological world. A lack of structure may also make it unclear as to the responsibilities of the individuals in an organisation, when it comes to driving and championing these types of change. In a framework proposed by a group of Hong Kong Polytechnic University researchers, a number of critical success factors (CSFs) for the adoption of enterprise resource planning (ERP) software were outlined \cite{ngai2008examining}. Amongst the outlined CSFs, exist a number of factors which may prove difficult to achieve for \emph{Start-up} stage NFPs, such as \emph{Business process/rules are well understood}, \emph{Empowered decision makers}, \emph{Company-wide commitment}, \emph{Organisational experience of IT or organisational change projects of a similar scale}, and \emph{Alignment between business strategy and IT strategy}, amongst others. And whilst this research does primarily pertain to the adoption of ERP software, which is quite large and complex by nature, it is still a relevant work in identifying some success factors which may prove to be more of a barrier for a less mature NFP.

There are also barriers to adopting technology that exist in a more foundational sense. Lack of general computer literacy, or the inability to recognise the potential that technology can bring to an organisation, may result in the option never even being considered viable or useful. And even if the option of shifting towards a more modern approach of business is considered, it may often fall outside the range of affordability, or the resources to implement change may not be available. These concerns are not new, and have long been recognised as a challenge for organisations, in their inability to be cognisant of the opportunities of which they are unaware \cite{benjamin1983information}. A 2003 article from the \emph{Harvard Business Review}, aptly titled \emph{Wanted: Chief Ignorance Officer}, recognises that steering away from the known to the \emph{terra incognitia}, or the unknown, is a requirement for innovation to occur \cite{gray2003wanted}. For an organisation that may not hold this ideology, technological adoption may never even be considered.

For technology to be successfully utilised by NFPs, particularly in connecting with their end users, these barriers must be considered, and methods to overcome them must be discovered. Further on in this review, we will discuss the incentives and potential pathways with which NFPs, their end users, and the surrounding ecosystem can assist in overcoming these challenges.

\section{Technology for those Experiencing Homelessness}

As recently as 2008, a study dishearteningly found that \emph{“for every person who thinks the homeless are not dangerous; another person thinks that they are”} \cite{donley2008perception}, illustrating the dichotomy of public perceptions when it comes to those experiencing homelessness. As well as these perceptions around safety, it is often more generally assumed that the lives of people experience homelessness are drastically different from those that are not. And whilst this is true in many regards, the differences in utilisation of technology are smaller than is often imagined \cite{le2008designs} \cite{pollio2013technology} \cite{rhoades2017no}. In understanding the current state of technology utilisation across the homeless population, it is also an interesting point of discussion to consider the historical context which leads to these misconceptions.

\subsection{Historical Technology Usage}

Prior to the ubiquity of owning and operating a personal technology device, such as a phone or a laptop, it can be assumed that those who were experiencing homelessness were technologically disconnected from the wider world similarly to those who were not having the same experience. With a reported 1\% of US adults having broadband access in their own home in the year 2000 \cite{smith_2017}, it is important to recognise that the technological connection gap between those who were experiencing homelessness, and those who were not, was once not as much of a pressing concern.

Earlier studies into technology use amongst those experiencing homelessness show that the utilisation of technology was traditionally more of an intentional and sustained effort, rather than a habitual part of everyday life. Those experiencing homelessness would often need to seek out a library, or some other form of public access, to be able to utilise the most rudimentary technological applications \cite{eyrich2011sheltered} \cite{le2008designs}. Whilst it was often found that the means of accessing and utilising technology were drastically different from those who were not disadvantaged, the reasons for using the technology was often not too disparate. Whether it was checking the weather, searching for jobs, or reading the news, the desires that were able to be met through technology have seemingly always been universal.

Whilst this universality in desired technology use has remained mostly unchanged over time, the universality of access has increased drastically since the times of these early studies. With an estimated 73\% of US adults now having broadband access at home \cite{smith_2017}, it must subsequently be acknowledged that many of these earlier sources must be considered primarily for their historical context rather than their current relevancy.

\subsection{Modern Technology Usage}

Along with an omnipresent increase in all forms of technological use in recent years, the rates of technology utilisation by people experiencing homelessness have also been found to be rapidly increasing \cite{rhoades2017no} \cite{eyrich2011sheltered} \cite{pollio2013technology}, particularly amongst younger demographics, or any other demographic that is already more likely to utilise technology across the wider population \cite{woelfer2010homeless}.

A 2014 Australian investigation surprisingly found that mobile phone ownership within the homeless population (95\%) actually exceeded the mobile phone ownership rate of the general Australian population (92\%) \cite{humphry2014importance}. This finding also held true, to an even greater extent in fact, when only smartphones were considered (77\% vs 64\%). The study also emphasises that these findings were \emph{“not indicative of their affordability but rather the degree of their importance and priority given to them by participants because of their essential role for ‘survival’ when no ready alternative was available”}. Whilst the rates of ownership were found to be encouraging, a number of challenges are still present when it comes to the utilisation of these devices. In particular, it was found that individuals may struggle to keep the device charged, or connected through a network provider. These findings were further supported by alternative studies \cite{raven2018mobile} \cite{woelfer2011artifacts}, and a number of recommendations, including development of low energy smartphones \cite{kinzer2014low}, have been proposed.

Technology is generally utilised by people experiencing homelessness in extremely similar ways as those who are not. e.g. social networking, checking the weather, searching for jobs, searching for services \cite{rhoades2017no} \cite{eyrich2011sheltered} \cite{woelfer2010homeless}. The high rate of mobile phone ownership has also been linked to certain activities which are often required to be undertaken by those experiencing homelessness, particular those related to welfare and support programs. For example programs that require participants to record actions such as job applications and job interviews are often best recorded using a mobile phone \cite{humphry2014importance}.

Studies have also shown that there is a strong belief in the homeless community that technology can be readily utilised to connect, empower, and educate others who find themselves temporarily within this community \cite{eyrich2011sheltered}. This connection enables those who can best empathise with someone experiencing homelessness to provide vital empathy and support in a time of need.

\subsection{Benefits of Technology}

As already identified, technology use amongst those experiencing homelessness is not only far more commonplace and widespread than usually assumed, but also the gateway to an incredible range of untapped opportunities aimed at benefiting these individuals. In fact, technology has been already been found to provide people experiencing homelessness with a vast array of benefits.

It has been proposed that mobile phone technology should form an essential component in efforts to improve the safety of young people experiencing homelessness \cite{woelfer2011improving}. The proposal recognises the intrinsic link between an individual's safety and their mobile phone, primarily as a result of the mobile phone being worn on the body. Specific design opportunities resulting from this proposal range from \emph{Support a homeless young person’s need to document abuse}, through to \emph{Support for safe, non-stigmatized access to infrastructure}. Further development in these areas could prove to be immensely beneficial for those looking to their mobile phones for an added layer of safety, particular those who are often found with limited layers of safety to begin with.

In parallel, it has also been shown that technology can be utilised and encouraged in order to reduce substance abuse amongst a similar population \cite{rice2011social}. One of the key focuses on the substance abuse study was the importance of social networking technology, and digital connection, that provide tangible benefits to the user.

Further, studies of the importance of technological connection for people experiencing homelessness show that there are significant mental health and wellbeing benefits to the individual \cite{rice2011social} \cite{eyrich2010mobile} \cite{roberson2010survival}. Reduction in the incidence of HIV infections, incidences of depression, and levels of addiction amongst those experiencing homelessness have all been recognised as potential outcomes from the connection provided by modern technology \cite{sala2014benefits}.

It is important to note that the benefits of technology outlined in this section of the review are intended to illustrate the benefits that may not be similarly apparent in the general population. However, it is still important to note the important benefits that the general population obtain from technology can also be realised by those experiencing homelessness, such as finding a job \cite{le2008feature}.

An increasing utilisation of technology amongst those experiencing homelessness, and the evidential benefits which this provides, results in a growing need for NFPs to more appropriately integrate these individuals into their already broadening technology strategies.

\subsection{Barriers to Success}

Whilst some form of access to modern technology has been shown to be almost universal in the population of those experiencing homelessness, there are still a number of barriers which have been found when compared to the general population. Some of these barriers have already been outlined, specifically those pertaining to battery charge or network connectivity, and a further set of challenges will be investigated later in this report. Not only will we review barriers experienced directly by those who are experiencing homelessness, but also by those who are developing technology specifically for this user group. The intersection of the barriers faced by both of these parties, evidently results in the disinterest, or lack of incentive, to develop and/or utilise such technologies.

When developing technology for a user group that has universal access to a mobile phone for example, it is easy for a developer to require the use of a native application, or multi-factor authentication, to interact with their platform. Inherently, these developers have made the decision that they are not interested in supporting the use case of an individual who does not have access to a mobile phone. The alternatives to this decision require the developer to either develop and maintain another technological form of interaction with their platform that does not require a mobile phone, or to allow for more traditional methods such as phone calls or brick and mortar establishments. Both of these alternatives are obviously more cumbersome and expensive for the organisation, and often will not be considered.

This predicament in designing for those who do not have universal access to mobile phones, can be extended to also consider any other similar requirement such as Wi-Fi, GPS, or phone credit. When designing technology with the intention of connecting those experiencing homelessness, this predicament will often not be overcome, and a solution will either be provided that is inaccessible by a large number within the target user base, or no solution will be provided at all.

Specific design considerations, aimed at combating many of these outlined barriers, will be further investigated later on in our review.

\section{Connecting to those Experiencing Homelessness}

The benefits that can be attained from people experiencing homelessness utilising technology are extremely evident; as are the benefits that NFPs stand to obtain by further embracing technology. In light of these observations, it becomes apparent that there exist many opportunities for NFPs to further benefit the lives of those experiencing homelessness; specifically, through an emphasis on technology designed specifically for the individuals they are trying to serve.

These opportunities present a pathway towards mutual benefit, for both the NFP and those experiencing homelessness. The NFP has the potential to benefit from technology through advancements in reporting and data collection, optimising and streamlining operational processes, and ultimately achieving their proposed mission through the betterment of those experiencing homelessness. As previously discussed, the benefit technology can provide to those experiencing homelessness are vast, and largely untapped at a large scale.

Efforts from NFPs to provide technology directly to those experiencing homelessness are not commonplace in Australia. The most notable effort thus far is undoubtedly Ask Izzy, developed by Melbourne based organisation Infoxchange. Ask Izzy acts as a service finder for those experiencing homelessness, and is an excellent example of technology being developed with the sole focus of improving the lives of those in need. An in-depth evaluation of Ask Izzy found it to be empowering for those who used it to find new information \cite{burrows2019evaluating}. The evaluation also uncovered a social component of the service, as it was found that users would often be accessing on behalf of another, or by service providers themselves.

Whilst Ask Izzy is an excellent example of technology being designed and implemented for the sole benefit of those in need, often due to experiencing homelessness, it is only one such example in what should seemingly be a cluttered landscape. There does exist alternative, often geographically constrained services, however these smaller services are often incomplete, unknown to those in need, too specific for widespread use, or otherwise insufficient \cite{gough2013improving} \cite{gough2016designing}.

In attempting to provide further examples to be discussed alongside Ask Izzy, this project aims to outline an approach to fill this gap in the technological connection between NFPs and those they are trying to support.

\subsection{Benefits for a Not-for-Profit}

An individual's experience with a NFP service, like an experience with any service, can be considered to be a journey, composing of a number of discrete phases. At the start of this journey, an individual is unaware of the existence of this service, and at the end of this journey they may be a regular user of the service, or even an advocate for the service in the wider community.

Initial discovery of a service is the first step for an individual, and this step can be increasingly simplified through an NFPs utilisation of technology. Whether it be through digital advertising, online news stories, listing in an online service directory, or otherwise, and individual will likely discover a service over the internet. Even in the case where an individual discovers a service through word-of-mouth, they will still likely validate and explore this information further by investigating an NFPs online presence. This online presence can potentially extend beyond the basic information and marketing of a service, to include extensions such as logistical information, terms of use, and testimonials. Examples of this can be seen at sites like \url{locations.orangesky.org.au}.

Upon discovering and subsequently committing to partaking in a particular service, an individual user's experience can be benefited greatly through technological systems aimed at simplifying logistical issues such as booking, cancellations, and scheduling, as well as informative issues such as contact and directional details for the service. An adjacent study into the digital experiences of users accessing service information online outlined that \emph{“interactions with service websites create opportunities for positive experiences that can foster trust and brand equity”} \cite{bilgihan2015applying}. During a service, a user may also benefit from technology that has been implemented to simplify and streamline the service delivery, or even to provide extra comfort and anonymity. It has also been shown in prior work that adequate utilisation of technology does not solely serve to enhance service delivery of an NFP, but also stands as a potential barrier to adoption by end users, in the case where the technology is inadequate \cite{walker2002technology}.

Beyond the participation in an NFPs service, individuals may wish to be provided with further information such as reports, referrals, and other opportunities which may be useful. NFPs may also to wish to follow up with their users to show appreciation, confirm the delivery of service, or to otherwise further engage the individual. All of these opportunities that exist beyond the point of participation are excellent examples of ways in which technology can augment the experience, and provide mutual benefit to all parties. These techniques are not unique to homelessness support services, and have been thoroughly studied in adjacent areas \cite{bovaird2007beyond}.

Not only are technological innovations relevant in the area of service delivery when it comes to connecting directly with end users. Technology may also be utilised to foster a community of users, NFPs, and other third parties who are all digitally connected. Through forums, social media channels, and other forms of communication, the connection between all parties surrounding an NFPs service may be connected in an ongoing capacity. The shared experiences which exist between these parties allow for immediate relevancy to be established between everyone involved, and hopefully allows for strong bonds to be formed throughout the surrounding community. NFPs should note that the \emph{Harvard Business Review} proposes that the most potential can be recognised from an organisation's online community when the community is controlled by the members themselves, in an environment created by the organisation to enable such a community to thrive \cite{fournier2009getting}.

\subsection{Risks for a Not-for-Profit}

Alongside the immense benefits which an NFP may obtain through the integration of technology into their operating strategy, there are also a number of risks which should be considered. These risks must form an integral part of the organisation's decision-making strategies when making decisions related to technology.

The misuse and subsequent public distrust, in the collection of individual's data has evolved quite rapidly in recent years. An organisation must always consider the ethical problems surrounding any data they wish to collect, as well as how they will ultimately store, analyse, and display this data. Incorrect or unethical considerations around individual data may result in an NFPs technology being seen as pervasive, indulgent, or unnecessary, amongst others. The resultant reputational and operational risk for an NFP must not be overlooked. A piece by Hirsch in 2014, foresaw many of the affects that increasing data velocity would have on the corporate world, particularly corporate reputation, insightfully stating that \emph{“consumers will appreciate companies that both take this area seriously and are willing to grapple with the implications”} \cite{hirsch2013corporate}. As NFPs invest in their technology strategies, this insight by Hirsch should remain a key consideration.

It is also commonly a core part of NFP services, particularly in the homelessness space, that anonymity is offered to all end users. Not only should any technological solution allow for the highest degree of anonymity to be retained by the end users, but it should also be implemented such that this anonymity is obvious to the end user, who will likely not have an enhanced technical understanding of the problem. If an end user is unable to feel anonymous through a service offering from these NFPs, then there is the increased potential for them to disengage, provide false information, or even tarnish the NFPs reputation by word-of-mouth \cite{gefen2003managing}. The ways in which different types and groups of users develop trust in a technical product is known to differ \cite{xu2014different}, and as such, a generic view of user trust should not be assumed. NFPs must ensure that care is taken to understand the antecedents to trusting their software in a manner that is tailored to their intended users. The technical problem behind retaining anonymity has become increasingly prominent recently with COVID-19 tracking apps, and a lot of literature exists on the topic \cite{rowe2020contact} \cite{tang2020privacy} \cite{cho2020contact}. These outlined best practices and methodologies should be considered in designing technology by these NFPs.

An organisation who chooses to embark upon the journey of delivering a technical product, must also acknowledge the commitment that they are making to maintaining, developing, and delivering the product for the lifetime of the solution. The ongoing cost of investing in technology development cannot be overlooked, and must be factored in financially. An explanatory piece of work on the software development life cycle outlines \emph{Poor maintenance} as one of the key reasons for low quality apps \cite{inukollu2014factors}. Inadequate resourcing towards the ongoing aspect of these projects may result in an unreliable, insecure, or outdated piece of technology; ultimately resulting in similar reputational and operational risks as the above issues. 

Whilst there to does exist a multitude of risks which must be considered, many of these are common across all industries, and therefore prior work can be heavily considered when pursuing these strategies.

\subsection{Prior Work by Not-for-Profits}

Although it has been acknowledged that there is not a large quantity of technology aimed directly at those experiencing homelessness, there has been some previous discussions and events aiming to improve this.

Hackathons are a staple in any industry utilising software engineering, described by Briscoe as \emph{“a problem-focused computer programming event”} \cite{briscoe2014digital}. Usually an event running for a day or more, dedicating a passionate group of developers, designers, and product people at a specific problem. This problem is often business focussed, or process focussed, however it can realistically be adjusted to pretty much anything people are willing to spend time on. As a result of the flexibility this format provides, it has been readily embraced multiple times in attempts to help solve the biggest problems around homelessness \cite{wilson2019beyond} \cite{linnell2014hack} \cite{pogavcar2016urban}. Whilst it was notably difficult at getting products from these hackathons to market, they have been recognised as a good opportunity to better understand the issues, and to further educate those involved on said issues.

\subsection{Considerations in Design}

In the design of any system, technological or otherwise, it is vital to make considerations in alignment with the desired or identified user groups. Applying this ideology to the design of technology for those experiencing homelessness provides interesting insights that need to be considered during the design process.

The nuance of the different groups within the homeless community is extremely complex. The homeless community is often thought of as a single demographic; whereas in reality, the homeless community is an incredibly diverse community with representation equally diverse to that of the general population \cite{abs2016census}.

A 1990 study targeting specifically women experiencing homelessness stated that due to the complexity of homelessness, development within the space \emph{“requires careful analysis of the unique features of identifiable groups within the homeless population”} \cite{hagen1990designing}. This careful analysis allows any development within the space to best benefit the intended group, or groups, of users. For example, a study aiming to understand and design around the needs of people experiencing both homelessness and depression found that supporting the immediate and future needs of the individuals was the most beneficial course of action for this particular group \cite{belcher1990needs}.

Designing around the general needs of a desired group within the homeless community also allows for design to be focused on the collective needs of the users, as outlined by Lehane \cite{lehane2010designing}. Lehane suggests that the correct user experience decisions can be made by treating every individual as part of a larger collaboration, and considering exclusively the factors binding these individuals. Ultimately designing not just for the user, but for the user's community of practice. In the case of people experiencing homelessness, it may be beneficial to identify not just the relevant user groups within the homeless community, but also to consider what binds these user groups, and the shared practices which they undertake.

\subsection{Accessibility}

As has been identified, the importance of designing new systems specifically for those experiencing homelessness has immense potential. However, there is also a lot of potential that can be realised by adjusting the way in which existing technology is developed.

Accessibility is a pertinent concern of any modern technology endeavour, the best efforts are made on websites, apps, and media to ensure that the content can be utilised by as many people as possible \cite{henry2014accessibility} \cite{brophy2007web}.

Nevertheless, a Google study found that there are some widespread accessibility pitfalls that often lead to making certain technologies inaccessible to those who are financially insecure \cite{sleeper2019tough}. These pitfalls usually fall outside the commonly considered accessibility requirements, mainly those pertaining web technologies such as HTML and CSS, indicating that there is an unconscious bias when it comes to the types of accessibility that are commonly accounted for. This study also suggest ways in which these pitfalls could be rectified, and encourages developers to consider the implications of these issues. More specifically the study outlines four key challenges that are faced by financially insecure users:

\begin{enumerate}
    \item \emph{Limited financial resources}, resulting in susceptibility to scams, difficulty withholding personal information, and difficulty in quickly recovering from an issue
    \item \emph{Limited access to reliable devices \& Internet}, resulting in the need use public or shared devices, or the inability to utilise multi-factor authentication methods
    \item \emph{Untrusted relationships}, resulting in devices being vulnerable to untrusted use
    \item \emph{Ongoing stress}, resulting in drastic response to security and privacy threats, and difficulty in taking decisive actions
\end{enumerate}

This distillation of some core challenges faced by this user group provides an excellent insight into the types of accessibility concerns which are often overlooked when developing for emerging, or otherwise under-represented user groups. It is critical that any software designed for those experiencing homelessness is cognisant of these issues.

Another study also highlighted the importance of including emerging technology users in design considerations now, rather than waiting for them to catch up technologically \cite{jones2017beyond}. By including the emerging technology users, potentially on lower end or older hardware, developers are ultimately able to engage, and design around, a previously untapped group of potential users.

Accounting for the above considerations when designing technology for those experiencing homelessness will ultimately result in a product that is empathetic, useful, and empowering for the user.

\subsection{Collaboration in Design}

During the design of any technology for those experiencing homelessness, it is important to collaborate with the eventual users. This collaboration can have many forms, and also many potential benefits.

A variety of co-design principles were validated in the homeless community through an SMS service project in London \cite{kwon2013co}. The project showed that co-design with a vulnerable user group can lead to a sense of shared ownership, and improved participation, of the end technological product.

Beyond the homeless community, it has been observed that collaboration with a diverse range of stakeholders will ultimately lead to a wider set of possible solutions being considered in almost any circumstance \cite{doberstein2016designing}. Further, it has been shown that co-design is not only influential on the possible solutions to a perceived problem, but can also influence and modify the understanding of the perceived problems themselves \cite{melles2012empower}.

One Canadian project even went as far as to put the entire responsibility of design and development of a mobile application for youth experiencing homelessness into the hands of youths experiencing homelessness \cite{buccieri2015empowering}. This project was aimed at empowering the youths, as well as harnessing their experiences and knowledge in a way that can be beneficial to others experiencing similar events.

In ensuring the involvement of as many end user groups as possible in the design process, it is evidently conducive to creating technology that accurately identifies, responds to, and empathises with the needs of these same groups.

\section{Promoting Not-for-Profit Technology}

In investigating the relationship with technology for both homelessness NFPs, and for those experiencing homelessness, it is extremely evident that individually there is immense benefit to be recognised through embracing modern technology. Our review has also seen that the intersection of these parties, and their relationships with technology, allow for potentially compounding mutual benefits to be obtained. 

NFPs that are able to be proactive in their embrace of technology will not only be able to provide better service delivery, have enhanced ability for outreach and engagement, and raise more funds, but they will be able to do this whilst simultaneously providing additional benefit to those experiencing homelessness. 

Those experiencing homelessness stand to benefit greatly from an NFP providing a technologically augmented service model that better caters to their needs, along with the ability to be more informed about the accessible services and therefore make more informed decisions around logistics and service participation. Ultimately, technology based solutions will enhance the connection between NFPs and those experiencing homelessness, leading to better outcomes for both parties as previously outlined.

\subsection{Existing Incentives}

It is also paramount for this review that there is consideration for the existing incentives for an NFP to invest in technology based solutions, particularly those connecting themselves directly to their end users. These incentives can be considered to exist in both a formal and an informal capacity.

Formally, NFPs are able to take part in many challenges, grants, or other such formalised funding processes to bring to life an idea or a project that will connect to their end users through technology. These processes are often run by large organisations with a philanthropic mission, with more and more of an emphasis on the utilisation of technology as the years go by. Australian NFPs are estimated to receive almost half (47\%) of their funding from the government, with the percentage on an individual NFP basis increasing as the size of the organisation increases \cite{acnc2020}.

Whilst these more formal processes have underpinned the Australian NFP sector for generations, a study by the University of Technology in Sydney recognises that this sector is \emph{“in transition”} \cite{logue2014emergence}. This transition leaves the formal model of NFP operation as just on possible model, amongst an extremely fluid set of models that shift the NFP into an entity within a wider \emph{“social economy”}. 

In alignment with this transition into informality, it may be considered by an NFP that a strong technology strategy and implementation will excite potential investors and ultimately result in greater funding. Strong adoption of technology within an NFP may be seen by certain donors as a strong indication that an organisation is wisely distributing their resources, and possesses a strong potential for scale. These observations by donors may result in enhanced funding opportunities, and subsequently the enhanced potential for an organisation to scale.

Another means of informal incentive for an NFP to benefit from technological investment is an extension of the earlier discussion around mutual benefit for the end users. An NFP that is able to more readily provide an excellent service, to an informed group of end users, could see an enhanced growth in reputation as a result. Reputation is incredibly important in the NFP space, as there is often competition for funding and growth potential within the industry. It should be noted that although this competition may seem to be counterproductive to the altruistic nature of the NFP sector, it is actually provably beneficial in a similar sense to healthy competition in the for-profit world \cite{philipson2009antitrust}. Following from the observation of the healthiness of competition, is the observation that this competition may subsequently drive an NFP towards technological progression and adoption, resulting in an ultimate benefit for the end user of the NFP's services.

\subsection{Future Avenues}

Whilst there does exist a variety of incentives for homelessness NFPs to invest in their own technological solutions, accompanied by an evidentially large number of donors interested in assisting financially, there is still the potential to better grow this area of technological connection to end users.

Many of the facets of an NFPs operations that can be augmented through technology remain the same, or similar, from organisation to organisation. Therefore, the opportunity for a standardised technological solution to be developed and shared within these facets is apparent. Financial backing by already interested donors may be utilised to support the ongoing development and delivery of technology solutions that may be shared across the NFP space. Providing a standardised, scalable, and affordable solution to everyone in the industry, without the requirement from large individual and bespoke solutions. Recent support for this ideology has been shown by Google, awarding the 2018 Google Impact Challenge \$1M prize to a software project aiming to standardise the volunteer management software used across the Australian NFP space \cite{osa2018}.

Adjacent to the idea of developing shared solutions to shared problems, it may also be beneficial for NFPs to invest in information sharing infrastructure that could more readily connect NFPs to each other, ultimately connecting those experiencing homelessness to a multitude of beneficial services and individuals. The ability for NFPs to share more information between each other would not only reduce the existing redundancies in data collection and analysis, but would also bring to light opportunities to collaborate more often and more effectively. Benson goes as far as challenging the sector in that there \emph{“is agreement that open data is worth it, now the question is how do we go about it?”} \cite{van2015open}.

Not only should information be shared by NFPs, to other NFPs, it should also be shared by other third parties to promote the technological success of certain projects or initiatives \cite{van2015open}. Promoting success across the industry is an excellent mechanism for encouraging others, and representing the potentials for success that do exist down the route of technological adoption or investment.

\section{Review Summary}

We conclude from our review of the literature that there are immense benefits to obtained through the greater emphasis of using technology to directly connect homelessness NFPs to those experiencing homelessness. We observe that the universality of access to modern technology within the homeless population has reached a critical mass at which it is now possible for NFPs to develop technology based solutions with their direct end users in mind. Also acknowledged are the existing barriers to entry, and potentially lacking incentives, for NFPs to willingly invest in this space.

Further analysis into the potential use cases of individual NFPs may provide a more detailed understanding of the barriers, incentives, implementations, and end result of technological adoption strategies within homelessness NFPs. It may also be beneficial to further study success stories within this space, in an effort to better understand the factors that enable the successful connection to those experiencing homelessness.

Alongside these areas for further research, we have identified the need to provide further incentive and awareness to organisations that may be in a position to benefit from a technological shift with their organisational strategy. Through sharing information, standardising solutions to common problems, and promoting successes within technology-focused NFPs, we believe that there is large potential for improvement in the lives and connectedness of those experiencing homelessness.

The remainder of this report should aim to further investigate the findings of this review, in an attempt to expand upon, further understand, and potentially validate or invalidate some concepts that have been uncovered.