\chapter{Results and Discussion}

As has already been outlined throughout the previous sections of this report, our intent was to perform a multi-faceted investigation into a wide range of factors impacting the overall success of an NFP utilising technology to connect directly to those experiencing homelessness. We have attempted to categorise the varying factors that we have investigated, and provide subsequent analysis of our results, extract our key findings, and ultimately put forth any recommendations for future work in the space, by both researchers and NFPs themselves.

Note that supporting information and transcriptions of all captured data can be seen in more detail in Appendix \ref{appendix:transcriptions}.

\section{Impacts of COVID-19}

We also note that the overall implementation of our project was hindered by a multitude of external factors, and as such, we have not been able to perform our research to the thorough extent which we had desired.

The impacts of COVID-19 generated heightened expectations and requirements for our initial ethics applications to be approved, resulting in a delayed commencement to our engagement with Orange Sky. The service model of Orange Sky was also heavily impacted by COVID-19, with all shifts across Australia being halted for a large period of time. This disruption in Orange Sky operations was subsequently also a disruption to our project, we were instead able to utilise this time to extend the depths of our initial review of the literature. Even upon recommencement of Orange Sky services, and thus our project, there were a number of health and safety restrictions enforced to provide all individuals present with a safe, socially distanced, environment. Whilst we are immensely appreciative of the severity with which Orange Sky treated their health and safety obligations to the community, we also acknowledge that this impacted some of our delivery strategies, as will be discussed.

Lastly, we acknowledge that the initial intent of our project was to both release and iterate on our application during the course of our study. This iteration was initially intended to not only occur based on our findings, but also on the findings and knowledge gained by the accompanying work of Almaraz and Pozzi. Due to the destructive impacts that COVID-19 has had on our adjoining time frames, it was unfortunately not possible for Almaraz and Pozzi to perform their work in a window that was conducive to informing our application's future development cycles. As a result of these difficulties, our application did not undergo any major iterations beyond the final prototype.

\section{Delivery}

In an effort to provide greater insight and guidance to NFPs who might attempt to implement similar applications to ours into their services, we ensured that observations were continuously made as to the effectiveness of different delivery strategies. Delivery success factors were observed to appear not just from the intentional variations in delivery methodology, as explicitly outlined earlier in this report, but also in the form of unintended emergent variations observed throughout our study.

\subsection{Network Effect}

Upon initial efforts to encourage and elicit participation in sharing stories via our application, it was evident that our application suffered from an almost causality-like dilemma. For us to be able to demonstrate to potential participants the effectiveness, or the outcome, of sharing their stories, we first needed to have already acquired some stories to illustrate this. We acknowledge that this phenomena, particularly used in the context of technology-based social applications, is often referred to as the \emph{network effect}; described by Shapiro as a situation where \emph{“the value of a product to one user depends on how many other users there are”} \cite{shapiro1998information}. Whilst this may seem like an obvious statement about our application, it was not foreseen, and thus proved to be an initial challenge in establishing our ability to readily collect many stories. The impact of this effect on our study extended the lead time, and minimised the time we were able to be in full delivery capacity. Shapiro's insight confirms that this growth pattern is to be expected in our application, stating that \emph{“technologies subject to strong network effects tend to exhibit long lead times followed by explosive growth”}. Unfortunately, the duration of our study did not allow our application to culminate in the prophesied explosive growth, however it was observed that the friction in the collection of stories reduced with each story collected. In future work, we encourage the consideration of this effect in any delivery strategies and proposed time frames; we also put forward the potential to use or gather an initial data set prior to application delivery, such that a portion of the long lead time can be mitigated.

\subsection{Formal Facilitation}

We also observed in our delivery strategies that accompanying a formal interview was an effective means of inciting participation in our application. Quantitatively, ???\% of our total stories collected came from this approach. Whilst not formally recorded due to the informal nature of some of our participation encouragement techniques, it is also noteworthy that the success rate in participation of this strategy was also substantially higher than other strategies. We believe that the existing trust and relationship that has been established, as well as the correlation with an individual's willingness to participate in a more formal interview process, are contributing factors to the success rate of this delivery strategy. We recommend this strategy to future efforts, particular those where high response rates are required.

\subsection{Volunteer Facilitated}

From a purely volumetric perspective, we find that the largest number of stories collected through our application come from the informal prompts to participate by existing Orange Sky volunteers, with a total of ??? (???\%) stories being shared through this strategy. Further, we saw that of the stories collected through this strategy, ???\% of these were shared by an individual to a volunteer with whom the individual had an existing relationship. Orange Sky encourages, and is rightfully proud of, the long term relationships which are formed between regular volunteers and regular visitors to their services. These existing relationships are based on trust, empathy, and non-judgement, all of which we believe are attributes conducive to an individual being willing to participate in our survey. A number of stories shared under the facilitation of a volunteer were also discovered to exhibit a conversational method of storytelling, for example the following recording is worth analysis:

\begin{drama}
    \Character{Kelsie (Volunteer)}{kels}
    \Character{Daniel (Friend)}{dan}

    \kelsspeaks: \emph{How long have you been coming down here Daniel?}
    \danspeaks: \emph{When yous first started.}
    \kelsspeaks: \emph{Yeah, and have you nicknamed everyone?}
    \danspeaks: \emph{Almost everyone.}
    \kelsspeaks: \emph{Tell me people's names and then their nicknames.}
    \danspeaks: \emph{Mel 1 \& 2, Soup, Mel 1.}
    \kelsspeaks: \emph{Stephen is \dots}
    \danspeaks: \emph{Bob the Builder. Joe is Jenga Joe, No Jenga Joe.}
    \kelsspeaks: \emph{And why is that?}
    \danspeaks: \emph{Because he doesn't bring the Jenga. Shark.}
    \kelsspeaks: \emph{What's shark's real name?}
    \danspeaks: \emph{Michael?}
    \kelsspeaks: \emph{Mark?}
    \danspeaks: \emph{Mark, yeah.}
    \kelsspeaks: \emph{And what about Sharna?}
    \danspeaks: \emph{Sharna Banana, and Little Red.}
    \kelsspeaks: \emph{Who's Little Red? Me? Kelsie?}
    \danspeaks: \emph{Kelsie!}
    \kelsspeaks: \emph{Yeah! Thanks Daniel!}
\end{drama}

The fundamental content that is being shared here by Daniel, specifically his passion for nicknaming all the Orange Sky volunteers, is not particularly of note. Daniel's content falls well within the bounds of the types of stories we expected individuals to share through our application. Significantly though, we witness an interesting method of prompting from Kelsie, to assist in extracting this story from Daniel. We recognise that whilst Daniel evidently had an interesting story that he wished to share, he potentially would not have been able to do so without the explicit assistance of another party. We acknowledge that whilst our application does not explicitly facilitate this conversational type of story capture, there are no apparent hindrances in allowing it to do so. Further, progress in this space should aim to harness the potential for conversational storytelling, and more generally the idea of multiple simultaneous participants. Story content, including Daniel's, will be discussed more broadly in a following section of this report.

\subsection{Device Limitations}

An inherent barrier to more widespread participation in our application was also found to be the nature of the device on which it was delivered. Whilst using the existing tablets, which are already seamlessly integrated into Orange Sky's service delivery models, enabled us to readily distribute and deliver our application, this methodology also proved to be limiting. The limitations imposed by this approach primarily arise from the inability for an individual to participate on their own device, on which they are likely far more comfortable and have a much larger window of accessibility. Compounding the lack of personal device access, is the implicit limitation of using a single instance of the application, resulting in mutual exclusivity between a story being shared and a story being listened to. Although it was not within the scope of this project, and we do not wish to have approached the device implementation differently, we propose that future work may enable higher rates of participation by distributing the application more broadly. It is also important for any future work to be aware of the risks and attack vectors which may be exposed when asking participants to install an application on their own personal devices. Distributing an application would also enable participation to occur outside the geographic and temporal bounds of an NFPs services, providing the opportunity for participants to take part in a time and space that is most comfortable for them.

\section{Technology}

Although the key focus areas of this project may lie beyond the details of any technical implementations, we still believe it to be a valuable discussion in sharing any findings we may have made in this area.

\subsection{Interface Design}

Primarily, we found that modern ubiquity across mobile application design principles was an excellent enabler in allowing participation from individuals with a broad degree of tech literacy. We did not observe any instances of the visual designs of our application to be a barrier to participation for any of our actual or potential participants. It is presumed that the standardisation in many of our interface elements provided users with a familiar experience, and did not require them to become familiarised with any elements foreign to their existing experiences. The familiarity of our design was enforced through the use of commonly used packages such as \emph{react-native-paper}, \emph{react-navigation}, \emph{react-native-community/slider}, \emph{react-native-community/slider}; we strongly recommend all of these packages in any future development, both for the simplicity they provide to the developer and for the familiarity they provide the user.

\subsection{Hardware Requirements}

We also find that the microphone, and subsequent playback quality was important in establishing trust in, and desire to utilise, our application. An individual reviewing a story they are about to share, or an individual listening to someone else's previously shared story, were anecdotally observed to have a greater experience if the microphone and speaker quality of the device were sufficient. This was observed due to the differing quality of devices upon which our project delivered our application. We also note that for certain devices and conditions, it is important to ensure the capture quality is sufficient before engaging participants in lengthy story sharing activities.

\section{Story Sharing}

Beyond the delivery strategies and technical implementations of our project, we further wish to analyse and understand the story sharing partaken through our application. We indent to provide interpretations and recommendations on the types of stories that have been shared, with an emphasis on the underlying motivations and contexts which encourage the respective individuals to do so.

\subsection{Story Taxonomy}

In qualitatively analysing the stories shared through our application, we propose that any given story may be categorised as one of the following:

\begin{itemize}
    \item \emph{Orange Sky Interaction}
    \item \emph{On the Record}
    \item \emph{Personal Anecdote}
\end{itemize}

To better analyse understand the content, motivation, and surrounding context of each of these story types, we will outline each of them in more detail, as well as providing a representative example.

\subsubsection{Orange Sky Interaction}

We identify these stories as those that are essentially a technologically augmented version of a regular interaction that an individual may have with Orange Sky volunteers or staff. For example, individuals sharing their broad appreciation or support for Orange Sky, or even sharing information about their relationship and utilisation of Orange Sky as a service. It was found that these stories generally came through the facilitation of an existing Orange Sky volunteer, and provided those experiencing homelessness with another means of interacting with the NFP's services. These stories were observed to be primarily jovial in nature, and seemingly facilitated a joyful interaction between NFP and the participant. We strongly recommend further investment in this space, by Orange Sky and the wider NFP space, as there exists significant evidence to suggest that the enablement of these interactions provides an enhanced experience for the users of these services.

Daniel's story, included in an above section of this report, is an excellent example of this type of story.

\subsubsection{On the Record}

An unexpected type of story that emerged from our analysis revolved around the seemingly novel experience that participant's saw in having the ability to simply record themselves. Whilst it may not be uncommon for individual's to have their voice transmitted digitally, usually over the phone or the internet, it is apparently more uncommon for voices to be recorded and stored more permanently. Having the ability for your voice recording to be distributed across not just space, but also time, encouraged participants to share things that they are passionate about, or things that they want other people to hear at some point in the future. These stories ranged from opinions on the nation's sporting landscape, to point-in-time updates about the current state. These point-in-time updates were seen to be used almost as a record-keeping mechanism, enabling future users of the application to reflect on a previous time. Lastly, we saw this type of story utilised to push forward an idea, or a call to action. By providing an application with no real limitations on content, we saw the emergence of a variety of different utilisation approaches from our participants. It is proposed that future work into this collaborative collection of user input, with limited restrictions, may provide diverse insight into the use cases which may materialise organically. We also commend our participants in their ingenuity in understanding the potential of our application beyond that of our own intentions.

The below example outlines a few aspects of this type of story, in its inclusion of its advertisement for a specific Orange Sky service, as well as a record of the point in time that it was recorded. Also, noteworthy is the desire to participate in our application as a volunteer, showing that the scope of participants out on shift was not limited solely to those utilising services.

\begin{drama}
    \Character{Kelsie (Volunteer)}{kels}
    \Character{Joe (Volunteer)}{joe}

    \kelsspeaks: So, Beenleigh's back Joseph?
    \joespeaks: Yes, Beenleigh's back every Thursday 6 til 8, showers and laundry.
    \kelsspeaks: And how long has it been since we were here?
    \joespeaks: Months, too long. Maybe \dots
    \kelsspeaks: How many months?
    \joespeaks: 4 months, 5 months.
    \kelsspeaks: And would you say it's pretty quiet compared to March?
    \joespeaks: Yes, very quiet, I'm looking forward to the full meal set-up coming back, and a few more friends coming back, but it's nice to see the regular faces again.
\end{drama}

\subsubsection{Personal Anecdote}

Lastly, we acknowledge the type of story that we incorrectly presumed would comprise almost all stories. These stories are those primarily shared as a way for an individual to express their own experiences, and to disseminate their feelings, learnings, and emotions that have come as a result of these experiences. It was found that these stories where the lengthiest of all story types, likely as a result of the details and context that an individual can provide when sharing their own experiences. As has already been acknowledged, the environment present in Orange Sky's current COVID-19 response operating model is not overly conducive to the collection of this type of story. By providing a more private or comfortable environment to our participants, it is presumed that the collection ability of this type of story would be enhanced.

The example we include here is simply a short representative example of this type of story. Appendix \ref{appendix:transcriptions} contains some far more lengthy and intimate examples of this type of story.

\begin{drama}
    \Character{Kelsie (Volunteer)}{kels}
    \Character{Daniel (Friend)}{dan}

    \kelsspeaks: Alright, Daniel, do you have a story for us?
    \danspeaks: Yes. The Orange Sky has been here since last year, until the COVID hit, and they're awesome.
    \kelsspeaks: And then what's been happening since we've been away?
    \danspeaks: I got a job.
    \kelsspeaks: Where'd you get a job at?
    \danspeaks: Near my place.
\end{drama}

\subsubsection{Taxonomic Limitations}

We emphasise that the taxonomy of our stories outlined above should only be considered as representative of our sample participants. We recognise that our delivery strategy may induce certain biases in our selection of participants, particularly towards those who have existing relationships with Orange Sky, or who have an existing willingness to participate altruistically in these types of projects. It should also be recognised that the environment in which our project operated was potentially not conducive to the sharing of more personal or intimate stories; a barrier which may be mitigated through the personal device distribution discussed earlier.