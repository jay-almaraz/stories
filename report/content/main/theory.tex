\chapter{Theory}

In our review of the existing literature we were able to identify a variety of potential benefits, perceived barriers, and future avenues for the technological connection between those experiencing homelessness and the surrounding NFPs. In this chapter of our investigation we aim to extract from our review the findings that we deem most valuable for the remainder of this project. Namely, we aim to bring some of our key findings into reality, and explore these findings in the real world. In our development and implementation of a DST solution we will aim to incorporate the existing DST literature into our other findings. Finally, our real world exploration will aim to further scrutinise the findings of our review, particularly those relating to NFPs developing and implementing technology to connect directly with end users.

\section{Those Experiencing Homelessness}

A key focus of our review revolved around the experience and adoption of technology for those experiencing homelessness. We aim to build upon this initial focus by developing real world empathy for the technological experiences of these individuals.

Our review was able to acknowledge that whilst ownership was not a key accessibility concern in participating in modern technological activities, there are a variety of other accessibility concerns which must be addressed. The general population stereotypically participate in a digital world where, without much exception, an individual is expected to be always contactable. This continuous expectation has some implicit requirements that may be difficult to meet for someone who is experiencing homelessness, such as requiring a device to be frequently charged, or for credit/data to be constantly available. These connectivity accessibility concerns were found to be amplified by similar concerns pertaining to individuals who may be financially insecure. Financially secure individuals were found to have difficulty participating in certain types of technology design, such as multi-factor authentications or assumed exclusive access to a device \cite{sleeper2019tough}. These accessibility concerns have been identified as crucial in any attempts at designing technology for these users.

Beyond access, our review also detailed the desires and common use cases exhibited by those experiencing technology. These use cases were found to align almost entirely with those of the general population, with the addition of desired use pertaining to support and service discovery, and welfare prompted record keeping. These findings suggest that there are no use cases that would be fit for the general population, that wouldn't also be appropriate and desirable by those experiencing homelessness.

A number of benefits of technology use by those experiencing homelessness were also found in the literature, with consistent use being linked to a variety of mental health benefits and deterrence programs. These benefits are also compounded by the benefits received through technology by the general population, such as finding a job, or increasing social connectivity through social media applications. In striving to acknowledge and embrace these benefits, our project will intend to develop a technical solution that capitalises and enables these benefits to materialise.

\section{Not-for-Profits}

The fiercely competitive nature of the modern NFP sector in the western world was found to be well documented in our review. As with any competitive industry, it is crucial for an organisation striving for success to be innovative in their strategy. Pathways towards innovation were also identified for NFPs, with many opportunities being underpinned by the rapid adoption of technology. Whether these opportunities were direct implementations of technology as a business model, or were enabled and supported by technology, it became clear that an NFP with a fledgling technology strategy is open for exploitation and supersedence by their competitors.

We found that internally NFPs are capable of adopting and exploiting technology to enhance their internal processes and operations, augment their data collection and reporting models, and also to widen their network and incubate their base of potential supporters. Externally, we saw that NFPs are able to provide a superior service to their end users by augmenting their delivery approach with technology. These augmentations were seen to be capable of elevating an individual's experience both before, during, and after the actual delivery of an NFPs service.

Our review also explores the incentives, risks, and barriers that NFPs may face in their journey towards technological adoption. Beyond the aforementioned benefits, NFPs may also stand to gain financially, reputationally, or through growth. These incentives are often supported by the investment appetites of third parties, and we found that technological innovation is an excellent approach for increasing this appetite. We further acknowledge that the adoption of new technology is not without risk, and empathise with NFPs in their common requirements for cost effective, low risk tolerance, strategies. In our review we outline best practices that NFPs can use to mitigate some of these risks, and we also explored some determining factors as to the organisational requirements necessary for attaining certain levels of technology adoption. These factors and mitigation methodologies will differ from NFP to NFP, and will be considered as such throughout the practical components of this project.

Whilst our review does touch on the potential avenues for increasing the incentive for NFPs to invest in technology strategies, we will not be bringing this into the remaining scope of this project.

\section{Intersection}

In addition to the thorough analysis of the two previously discussed parties, our review ultimately examines their intersection; particularly with emphasis on the role in which technology plays in connecting them. Building on the identified benefits that can be experienced with technology by both parties in isolation, we showed the success of existing efforts in achieving benefits in cooperation. We found that there were examples, albeit limited, of technology being successfully developed with the primary focus on those experiencing homelessness. It was also identified that there was interest and potential to further this space, previously explored through a number of hackathon events aimed at solving problems in the homelessness space.

The literature also clearly highlighted the considerations in design and accessibility that must be made when aiming to establish a technological link between NFPs and their end users. Further efforts in this project will ensure that the shared practices which are undertaken by those experiencing homelessness are considered foremost when discussing user groups. The accessibility considerations outlined above for those experiencing homelessness have been shown to be crucial if an NFP is to design a technology-based solution for their end users.

Lastly, we have found that collaboration in the design process of software connecting these two parties can be pivotal in developing an optimal solution. Co-design has been shown to instil a sense of shared ownership, and subsequently improve eventual participation in a project. Collaborating with a diverse set of stakeholders was also found to be beneficial beyond solving individual problems, extending to the diversification of problem identification in the first instance.

Building on our understanding of the technological intersection of these two parties, based on the existing literature, we will continue to develop this knowledge through practical implementation in this space.

\section{Digital Storytelling}

In attempting to identify a concrete implementation methodology which may be utilised in exploring the intersection of those experiencing homelessness and NFPs, our review identifies DST as a viable candidate. We saw that studies have demonstrably shown DST can benefit those experiencing homelessness, in a complementary manner to the benefits that were previously identified from general technological utilisation. Through educational and emotional engagement with DST implementations, those experiencing homelessness may realise personal growth, learning outcomes, or an increased sense of connection, among other identified benefits.

It was evident from our extensive review of the DST literature that the methodology is immediately applicable to the technological implementation we look to explore in this report. The method's applicability is derived not just from its proven benefits for those who are experiencing homelessness or are otherwise socially excluded, but also from its alignment with the desired outcomes for the end users of supporting NFPs.

This report will continue to explore the implementation details of DST in this space, and will ultimately develop and experiment with a DST application towards the end of this project.