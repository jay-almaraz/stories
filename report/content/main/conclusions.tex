\chapter{Conclusions}

After reviewing and distilling the existing literature, and subsequently heading out into the real-world with an established NFP, our study has been able to observe, uncover, and recommend a number of a key points of knowledge within this field. Despite being hindered by the presence of COVID-19, as well as some of our own oversights, we are pleased to present a number of concrete findings, as well as the potential foundations of future work.

\section{Findings}

By undertaking a generalised approach towards our investigation into the real-world implementation of a DST-based approach to connecting NFPs to those experiencing homelessness, we were able to produce a broad range of findings. These findings built upon our review of the existing literature, as well as our own interpretations of the theory underpinning our work. All findings and recommendations have been outlined more immediately in other parts of this report, however we intend to aggregate the broader discoveries of our work in this section.

Initially we set out to investigate the impacts of the evolving technological and societal landscape on those experiencing homelessness, particularly with respect to the connections between these individuals and the supporting NFPs. Our investigation aimed to validate and further investigate key theories through practical field work, as well as to develop increased empathy in the area. The key areas in which we claim to have made findings are \emph{technology delivery strategies for NFPs}, \emph{design principles when targeting those experiencing homelessness}, and \emph{the potential for user defined purpose in NFP technology}.

\subsection{Technology Delivery Strategies for NFPs}

We have recognised a number of factors that may lead to successful delivery strategies for NFPs to integrate technology into their existing service and connection models. 

Firstly, we reveal that the \emph{network effect} must be considered when initially delivering technology where the value of a user's experience is intrinsically proportional to the number of users. In mitigating this effect, we suggest organisations invest in a strategy that will allow for an initial data set, or user base, to be cultivated prior to commencing delivery. 

Subsequently, we show evidence that suggests a key barrier to participation in an application such as ours can be mitigated by careful consideration as to the ways in which participation is elicited and facilitated. Our work found that augmenting an existing formal process with technology had the highest rate of success, and we recommend this as an excellent approach when high response rate is desired. It has also been shown that informal participation in our application was most successful when an existing relationship was already present between the facilitator and the participant. It is encouraged that NFPs utilise their existing relationships when delivering similar technologies, particularly in the initial phases.

Lastly, we propose that our selection of a single shared device, available only on an Orange Sky shift and requiring manual facilitation, was not conducive to receiving a high volume of participants. However, we do confirm that this approach allows for tight control over parameters and evaluation, attributes that are important in a study like ours. If possible, based on the risk appetite and the nature of the technology, we suggest NFPs strongly consider the option for distributing technology directly to the devices of their end users.

\subsection{Design Principles for those Experiencing Homelessness}

As was recognised in our review of the literature, a proactive effort is required in designing technology for marginalised user groups, particularly those who are financially insecure or experiencing homelessness. Our implementation shows that through utilisation of existing standards, principles, and design libraries, it is possible to provide a simple and effective experience to all potential users. We also suggest that familiarity is important in this type of design, particularly when looking at a shared device model such as the one implemented in our project. Users of a shared device may have limited access time, and therefore immediate familiarity plays a pivotal role in their ability to fully understand the usage and functionality of the application. It is recommended that any design by NFPs, targeting a similar group of end users, should be performed with familiarity and usability as the highest priority.

\subsection{Potential for User Defined Purpose in NFP Technology}

Our most unexpected finding comes from the ingenuity shown by our participants, in their alternate interpretations of the benefits of using our application. Although our initial intention was to provide an application that allowed users to share and consume stories throughout the community surrounding an NFP, we unintentionally produced an application that allowed sharing of arbitrary thought, ideas, and any other content our participants might imagine. We recognise that this deviation from our intended usage actually highlights a serendipitous finding in our work. This finding relates to the mechanisms with which individual participants can find their own purpose in a technological platform, given the platform provides enough freedom to do so. In discussing our results, we suggested that further exploration of the wider benefits that user's may discover through a voice sharing platform could prove to be insightful. NFPs looking to implement similarly social technologies may also benefit from limiting the restrictions that are imposed, in an effort to allow users to define their own purpose, and ultimately realise their own benefits, from participation.

\section{Future Work}

This study has endeavoured to build on existing literature and empirical evidence to highlight a technological approach to connecting people experiencing homelessness and NFPs that support them. It serves as a preliminary investigation and offers opportunities for future studies to investigate and develop further. Throughout our report we have made our best efforts to provide advice on future work as it was deemed relevant. Generally speaking, there were some key themes for proposed future studies which have emerged throughout the construction of this report.

Whilst outlining our methodology for this project, specifically our delivery strategy, we acknowledged that only a small subset of all potential delivery strategies were to be explored. We encourage the undertaking of further work in an attempt to analyse the potential of delivery strategies that were not covered by our work. In particular, we place emphasis on our presumed potential in the delivery of similar applications by NFPs directly to the device's of those experiencing homelessness.

In any future studies following similar delivery strategies to those utilise din this project, we emphasise the importance of obtaining a larger sample size to more accurately analyse the delivery of the application. We also suggest the collection of accompanying qualitative data from participant individuals and NFPs regarding their experience. This qualitative data will enable a more rigid and structured analysis of the impacts of such an application on the real-world ecosystem in which it is delivered.

More specifically, we encourage Orange Sky and adjacent service to invest further into our application, or one similar in nature. We believe that by providing an Android application like ours, there is the potential to further expand upon the functionality and use cases that can be realised in this environment. We also encourage these NFPs to further instil technological connection into their service delivery strategies, and to invest in the mutual benefits that we have shown evidence for.

From the outset, our project aimed to analyse specifically the NFP sector surrounding those experiencing homelessness, and we therefore acknowledge the associated constraints on the relevancy of our work more broadly. Whilst we do believe that holistically our findings would be generally applicable in adjacent NFP sectors, we recommend further work is performed to test this assumption.