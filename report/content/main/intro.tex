\chapter{Introduction}

In a rapidly evolving technological and societal landscape, it is of increasing importance that no demographics, sectors of industry, or geographical regions are left behind. In this report we aim to investigate the impacts of this evolving landscape on one of society's most marginalised demographics, in those experiencing homelessness. We also intend to extend our investigation to include the ecosystem surrounding this community, specifically the supporting not-for-profit (NFP) organisations that provide invaluable services to these individuals. Emphasis will be placed on the role of technology when considered as a vital mechanism in connecting NFP services to the individuals whom they intend to serve. We strongly believe that a thorough understanding of technology's role in this ecosystem will result in meaningful recommendations and pathways towards increased mutual benefits in the space. This investigation also hopes to perform practical research in the field, in an effort to validate and build on existing empirical evidence, as well as to develop increased empathy in the area.

In acknowledgement of the immense scale of homelessness when viewed globally, with an estimated 150 million people currently experiencing homelessness \cite{chamie_2017}, as well as the inherent disparity across cultural and geographical boundaries, our investigation will primarily focus on the western world. Further, narrowing the scope, particular emphasis will be placed upon those experiencing homelessness, and the supporting NFPs, across Australia and New Zealand. NFPs that will be considered in this investigation will usually be limited to those whose core mission specifically relates to supporting those experiencing homelessness, or similarly marginalised individuals. In a further effort to ensure that the scope of this investigation is well bounded, prior work relating to technology will first be validated for its recency and current relevancy; acknowledging how rapidly certain technology-based studies may have become dated in recent times.

Overall, we wish to garner a deep understanding of the machinations of the Australian and New Zealand NFP sector, specifically the strategies in which modern technology plays a pivotal role in connecting to those experiencing homelessness. Regarding the implementation of such technology, we intend to utilise digital storytelling (DST) methodologies to exemplify one potential approach to connection in this space. Ideally, this understanding will result in new considerations that can be utilised for the mutual benefit of both NFPs, and their end users. Specifically, we aim to answer the question of \emph{“How can homelessness NFPs utilise DST technology in connecting with those who use their services?”}.

Structurally, we will begin by performing a detailed review of the existing literature in this area. This review will be followed by the design and development of a prototypical DST application. Our process will culminate in the attempted delivery of this application to those experiencing homelessness within the operational reality of an existing NFP's services. Ultimately, we will bring forth any findings and concluding thoughts that have arisen throughout our review, our design, and our delivery.